% \addtocounter{chapter}{1}\chapter*{}
% \markright{\MakeUppercase{\romannumeral\thechapter}}
% \addcontentsline{toc}{chapter}{~~~~~~~~}

\hypersetup{pdfstartview={Fit}, pdftitle={How to Live on 24 Hours a
Day}, pdfauthor={Arnold Bennett, LaTeX typeset by Ruben Berenguel},
pdfsubject={Time Management}, pdfkeywords={Time management, Conduct of
Life}, pdfpagelayout={TwoPageRight}}


\thispagestyle{empty}
\begin{center} 
  {
    \MakeUppercase{\fontsize{1cm}{1cm}\selectfont{How to Live on}}\\
\vspace*{0.3em}
    \textit{\MakeUppercase{\fontsize{1.1cm}{1.1cm}\selectfont{Twenty-Four}}}\\
\vspace*{0.3em}
    \MakeUppercase{\fontsize{1cm}{1cm}\selectfont{Hours a Day}}
    }\\
  \vspace*{1em}
  {
    \color{darkgray}\MakeUppercase{\fontsize{0.5cm}{0.5cm}\selectfont{Arnold Bennett}
    }}
\end{center}
\newpage
\vspace*{3cm}
\newpage
\color{black}
\thispagestyle{empty}
\begin{center}
  LaTeX formatted by Ruben Berenguel,\\
  \href{http://www.mostlymaths.net}{mostlymaths.net}, 2010.
\end{center}
\vspace*{1em}
{This is the Project Gutenberg eBook, How to Live on 24 Hours a Day, by Arnold
  Bennett.} 

\footnotesize\vspace*{1em} This eBook is for the use of anyone anywhere at
no cost and with almost no restrictions whatsoever. You may copy it,
give it away or re-use it under the terms of the Project Gutenberg
License included with this eBook or online at
\href{http://www.gutenberf.org}{www.gutenberg.org} \vspace*{1em}

\noindent\textbf{Title:} How to Live on 24 Hours a Day\\
\textbf{Author:} Arnold Bennett\\
\textbf{Posting Date:} August 23, 2008  [eBook Number 2274]\\
\textbf{Release Date:} August, 2000\\
\textbf{Language:} English\\
\textbf{Produced by:} Tony Adam\\
\textbf{HTML version by:} Al Haines\\
\vspace*{1em} 

\noindent The original ASCII document and all associated files of
various formats can be found in:\\
\href{http://www.gutenberg.org/2/2/7/2274/}{http://www.gutenberg.org/2/2/7/2274/}
\clearpage

\thispagestyle{ruled}
\pagestyle{ruled}

\setlength{\parskip}{0em}
\tableofcontents*

\setlength{\parskip}{0em}

\clearpage

\normalsize

\setcounter{chapter}{1}
\chapter*{Preface to This Edition}
\markright{\MakeUppercase{\romannumeral\thechapter} Preface to This Edition}
\addcontentsline{toc}{chapter}{~~~~~~~~Preface to This Edition}

This preface, though placed at the beginning, as a preface must be,
should be read at the end of the book.

I have received a large amount of correspondence concerning this small
work, and many reviews of it--some of them nearly as long as the book
itself--have been printed.  But scarcely any of the comment has been
adverse.  Some people have objected to a frivolity of tone; but as the
tone is not, in my opinion, at all frivolous, this objection did not
impress me; and had no weightier reproach been put forward I might
almost have been persuaded that the volume was flawless!  A more
serious stricture has, however, been offered--not in the press, but by
sundry obviously sincere correspondents--and I must deal with it.  A
reference to page 43 will show that I anticipated and feared this
disapprobation.  The sentence against which protests have been made is
as follows:--"In the majority of instances he [the typical man] does
not precisely feel a passion for his business; at best he does not
dislike it.  He begins his business functions with some reluctance, as
late as he can, and he ends them with joy, as early as he can. And his
engines, while he is engaged in his business, are seldom at their full
'h.p.'"

I am assured, in accents of unmistakable sincerity, that there are many
business men--not merely those in high positions or with fine
prospects, but modest subordinates with no hope of ever being much
better off--who do enjoy their business functions, who do not shirk
them, who do not arrive at the office as late as possible and depart as
early as possible, who, in a word, put the whole of their force into
their day's work and are genuinely fatigued at the end thereof.

I am ready to believe it.  I do believe it.  I know it.  I always knew
it.  Both in London and in the provinces it has been my lot to spend
long years in subordinate situations of business; and the fact did not
escape me that a certain proportion of my peers showed what amounted to
an honest passion for their duties, and that while engaged in those
duties they were really \textit{living} to the fullest extent of which they
were capable.  But I remain convinced that these fortunate and happy
individuals (happier perhaps than they guessed) did not and do not
constitute a majority, or anything like a majority.  I remain convinced
that the majority of decent average conscientious men of business (men
with aspirations and ideals) do not as a rule go home of a night
genuinely tired.  I remain convinced that they put not as much but as
little of themselves as they conscientiously can into the earning of a
livelihood, and that their vocation bores rather than interests them.

Nevertheless, I admit that the minority is of sufficient importance to
merit attention, and that I ought not to have ignored it so completely
as I did do.  The whole difficulty of the hard-working minority was put
in a single colloquial sentence by one of my correspondents.  He wrote:
"I am just as keen as anyone on doing something to 'exceed my
programme,' but allow me to tell you that when I get home at six thirty
p.m. I am not anything like so fresh as you seem to imagine."

Now I must point out that the case of the minority, who throw
themselves with passion and gusto into their daily business task, is
infinitely less deplorable than the case of the majority, who go
half-heartedly and feebly through their official day.  The former are
less in need of advice "how to live."  At any rate during their
official day of, say, eight hours they are really alive; their engines
are giving the full indicated "h.p."  The other eight working hours of
their day may be badly organised, or even frittered away; but it is
less disastrous to waste eight hours a day than sixteen hours a day; it
is better to have lived a bit than never to have lived at all. The real
tragedy is the tragedy of the man who is braced to effort neither in
the office nor out of it, and to this man this book is primarily
addressed.  "But," says the other and more fortunate man, "although my
ordinary programme is bigger than his, I want to exceed my programme
too!  I am living a bit; I want to live more. But I really can't do
another day's work on the top of my official day."

The fact is, I, the author, ought to have foreseen that I should appeal
most strongly to those who already had an interest in existence.  It is
always the man who has tasted life who demands more of it.  And it is
always the man who never gets out of bed who is the most difficult to
rouse.

Well, you of the minority, let us assume that the intensity of your
daily money-getting will not allow you to carry out quite all the
suggestions in the following pages.  Some of the suggestions may yet
stand.  I admit that you may not be able to use the time spent on the
journey home at night; but the suggestion for the journey to the office
in the morning is as practicable for you as for anybody. And that
weekly interval of forty hours, from Saturday to Monday, is yours just
as much as the other man's, though a slight accumulation of fatigue may
prevent you from employing the whole of your "h.p." upon it.  There
remains, then, the important portion of the three or more evenings a
week.  You tell me flatly that you are too tired to do anything outside
your programme at night.  In reply to which I tell you flatly that if
your ordinary day's work is thus exhausting, then the balance of your
life is wrong and must be adjusted.  A man's powers ought not to be
monopolised by his ordinary day's work. What, then, is to be done?

The obvious thing to do is to circumvent your ardour for your ordinary
day's work by a ruse. Employ your engines in something beyond the
programme before, and not after, you employ them on the programme
itself.  Briefly, get up earlier in the morning.  You say you cannot.
You say it is impossible for you to go earlier to bed of a night--to do
so would upset the entire household.  I do not think it is quite
impossible to go to bed earlier at night.  I think that if you persist
in rising earlier, and the consequence is insufficiency of sleep, you
will soon find a way of going to bed earlier.  But my impression is
that the consequences of rising earlier will not be an insufficiency of
sleep.  My impression, growing stronger every year, is that sleep is
partly a matter of habit--and of slackness.  I am convinced that most
people sleep as long as they do because they are at a loss for any
other diversion. How much sleep do you think is daily obtained by the
powerful healthy man who daily rattles up your street in charge of
Carter Patterson's van?  I have consulted a doctor on this point.  He
is a doctor who for twenty-four years has had a large general practice
in a large flourishing suburb of London, inhabited by exactly such
people as you and me.  He is a curt man, and his answer was curt:

"Most people sleep themselves stupid."

He went on to give his opinion that nine men out of ten would have
better health and more fun out of life if they spent less time in bed.

Other doctors have confirmed this judgment, which, of course, does not
apply to growing youths.

Rise an hour, an hour and a half, or even two hours earlier; and--if
you must--retire earlier when you can.  In the matter of exceeding
programmes, you will accomplish as much in one morning hour as in two
evening hours.  "But," you say, "I couldn't begin without some food,
and servants."  Surely, my dear sir, in an age when an excellent
spirit-lamp (including a saucepan) can be bought for less than a
shilling, you are not going to allow your highest welfare to depend
upon the precarious immediate co-operation of a fellow creature!
Instruct the fellow creature, whoever she may be, at night.  Tell her
to put a tray in a suitable position over night. On that tray two
biscuits, a cup and saucer, a box of matches and a spirit-lamp; on the
lamp, the saucepan; on the saucepan, the lid--but turned the wrong way
up; on the reversed lid, the small teapot, containing a minute quantity
of tea leaves.  You will then have to strike a match--that is all.  In
three minutes the water boils, and you pour it into the teapot (which
is already warm).  In three more minutes the tea is infused.  You can
begin your day while drinking it.  These details may seem trivial to
the foolish, but to the thoughtful they will not seem trivial.  The
proper, wise balancing of one's whole life may depend upon the
feasibility of a cup of tea at an unusual hour.\\
\vspace*{1em}

A. B.

\addtocounter{chapter}{1}\chapter*{The Daily Miracle}
\markright{\MakeUppercase{\romannumeral\thechapter} The Daily Miracle}
\addcontentsline{toc}{chapter}{~~~~~~~~The Daily Miracle}

"Yes, he's one of those men that don't know how to manage. Good
situation.  Regular income.  Quite enough for luxuries as well as
needs.  Not really extravagant.  And yet the fellow's always in
difficulties.  Somehow he gets nothing out of his money.  Excellent
flat--half empty!  Always looks as if he'd had the brokers in.  New
suit--old hat!  Magnificent necktie--baggy trousers!  Asks you to
dinner:  cut glass--bad mutton, or Turkish coffee--cracked cup!  He
can't understand it.  Explanation simply is that he fritters his income
away.  Wish I had the half of it! I'd  show him--"

So we have most of us criticised, at one time or another, in our
superior way.

We are nearly all chancellors of the exchequer:  it is the pride of the
moment.  Newspapers are full of articles explaining how to live on
such-and-such a sum, and these articles provoke a correspondence whose
violence proves the interest they excite.  Recently, in a daily organ,
a battle raged round the question whether a woman can exist nicely in
the country on L85 a year.  I have seen an essay, "How to live on eight
shillings a week."  But I have never seen an essay, "How to live on
twenty-four hours a day."  Yet it has been said that time is money.
That proverb understates the case.  Time is a great deal more than
money.  If you have time you can obtain money--usually.  But though you
have the wealth of a cloak-room attendant at the Carlton Hotel, you
cannot buy yourself a minute more time than I have, or the cat by the
fire has.


Philosophers have explained space.  They have not explained time. It is
the inexplicable raw material of everything.  With it, all is possible;
without it, nothing.  The supply of time is truly a daily miracle, an
affair genuinely astonishing when one examines it.  You wake up in the
morning, and lo! your purse is magically filled with twenty-four hours
of the unmanufactured tissue of the universe of your life!  It is
yours.  It is the most precious of possessions.  A highly singular
commodity, showered upon you in a manner as singular as the commodity
itself!

For remark!  No one can take it from you.  It is unstealable.  And no
one receives either more or less than you receive.

Talk about an ideal democracy!  In the realm of time there is no
aristocracy of wealth, and no aristocracy of intellect.  Genius is
never rewarded by even an extra hour a day.  And there is no
punishment.  Waste your infinitely precious commodity as much as you
will, and the supply will never be withheld from you.  No mysterious
power will say:--"This man is a fool, if not a knave.  He does not
deserve time; he shall be cut off at the meter." It is more certain
than consols, and payment of income is not affected by Sundays.
Moreover, you cannot draw on the future.  Impossible to get into debt!
You can only waste the passing moment.  You cannot waste to-morrow; it
is kept for you.  You cannot waste the next hour; it is kept for you.

I said the affair was a miracle.  Is it not?

You have to live on this twenty-four hours of daily time.  Out of it
you have to spin health, pleasure, money, content, respect, and the
evolution of your immortal soul.  Its right use, its most effective
use, is a matter of the highest urgency and of the most thrilling
actuality.  All depends on that.  Your happiness--the elusive prize
that you are all clutching for, my friends!--depends on that. Strange
that the newspapers, so enterprising and up-to-date as they are, are
not full of "How to live on a given income of time," instead of "How to
live on a given income of money"!  Money is far commoner than time.
When one reflects, one perceives that money is just about the commonest
thing there is.  It encumbers the earth in gross heaps.

If one can't contrive to live on a certain income of money, one earns a
little more--or steals it, or advertises for it.  One doesn't
necessarily muddle one's life because one can't quite manage on a
thousand pounds a year; one braces the muscles and makes it guineas,
and balances the budget.  But if one cannot arrange that an income of
twenty-four hours a day shall exactly cover all proper items of
expenditure, one does muddle one's life definitely.  The supply of
time, though gloriously regular, is cruelly restricted.


Which of us lives on twenty-four hours a day?  And when I say "lives,"
I do not mean exists, nor "muddles through."  Which of us is free from
that uneasy feeling that the "great spending departments" of his daily
life are not managed as they ought to be? Which of us is quite sure
that his fine suit is not surmounted by a shameful hat, or that in
attending to the crockery he has forgotten the quality of the food?
Which of us is not saying to himself--which of us has not been saying
to himself all his life:  "I shall alter that when I have a little more
time"?

We never shall have any more time.  We have, and we have always had,
all the time there is.  It is the realisation of this profound and
neglected truth (which, by the way, I have not discovered) that has led
me to the minute practical examination of daily time-expenditure.



\addtocounter{chapter}{1}\chapter*{The Desire to Exceed One's Programme}
\markright{\MakeUppercase{\romannumeral\thechapter} The Desire to Exceed One's Programme}
\addcontentsline{toc}{chapter}{~~~~~~~~The Desire to Exceed One's Programme}

"But," someone may remark, with the English disregard of everything
except the point, "what is he driving at with his twenty-four hours a
day? I have no difficulty in living on twenty-four hours a day.  I do
all that I want to do, and still find time to go in for newspaper
competitions.  Surely it is a simple affair, knowing that one has only
twenty-four hours a day, to content one's self with twenty-four hours a
day!"

To you, my dear sir, I present my excuses and apologies.  You are
precisely the man that I have been wishing to meet for about forty
years.  Will you kindly send me your name and address, and state your
charge for telling me how you do it?  Instead of me talking to you, you
ought to be talking to me. Please come forward.  That you exist, I am
convinced, and that I have not yet encountered you is my loss.
Meanwhile, until you appear, I will continue to chat with my companions
in distress--that innumerable band of souls who are haunted, more or
less painfully, by the feeling that the years slip by, and slip by, and
slip by, and that they have not yet been able to get their lives into
proper working order.

If we analyse that feeling,  we shall perceive it to be, primarily, one
of uneasiness, of expectation, of looking forward, of aspiration.  It
is a source of constant discomfort, for it behaves like a skeleton at
the feast of all our enjoyments.  We go to the theatre and laugh; but
between the acts it raises a skinny finger at us.  We rush violently
for the last train, and while we are cooling a long age on the platform
waiting for the last train, it promenades its bones up and down by our
side and inquires:  "O man, what hast thou done with thy youth?  What
art thou doing with thine age?"  You may urge that this feeling of
continuous looking forward, of aspiration, is part of life itself, and
inseparable from life itself.  True!

But there are degrees.  A man may desire to go to Mecca.  His
conscience tells him that he ought to go to Mecca.  He fares forth,
either by the aid of Cook's, or unassisted; he may probably never reach
Mecca; he may drown before he gets to Port Said; he may perish
ingloriously on the coast of the Red Sea; his desire may remain
eternally frustrate.  Unfulfilled aspiration may always trouble him.
But he will not be tormented in the same way as the man who, desiring
to reach Mecca, and harried by the desire to reach Mecca, never leaves
Brixton.

It is something to have left Brixton.  Most of us have not left
Brixton.  We have not even taken a cab to Ludgate Circus and inquired
from Cook's the price of a conducted tour.  And our excuse to ourselves
is that there are only twenty-four hours in the day.

If we further analyse our vague, uneasy aspiration, we shall, I think,
see that it springs from a fixed idea that we ought to do something in
addition to those things which we are loyally and morally obliged to
do.  We are obliged, by various codes written and unwritten, to
maintain ourselves and our families (if any) in health and comfort, to
pay our debts, to save, to increase our prosperity by increasing our
efficiency.  A task sufficiently difficult!  A task which very few of
us achieve!  A task often beyond our skill! Yet, if we succeed in it,
as we sometimes do, we are not satisfied; the skeleton is still with us.

And even when we realise that the task is beyond our skill, that our
powers cannot cope with it, we feel that we should be less discontented
if we gave to our powers, already overtaxed, something still further to
do.

And such is, indeed, the fact.  The wish to accomplish something
outside their formal programme is common to all men who in the course
of evolution have risen past a certain level.

Until an effort is made to satisfy that wish, the sense of uneasy
waiting for something to start which has not started will remain to
disturb the peace of the soul. That wish has been called by many names.
It is one form of the universal desire for knowledge.  And it is so
strong that men whose whole lives have been given to the systematic
acquirement of knowledge have been driven by it to overstep the limits
of their programme in search of still more knowledge.  Even Herbert
Spencer, in my opinion the greatest mind that ever lived, was often
forced by it into agreeable little backwaters of inquiry.

I imagine that in the majority of people who are conscious of the wish
to live--that is to say, people who have intellectual curiosity--the
aspiration to exceed formal programmes takes a literary shape.  They
would like to embark on a course of reading. Decidedly the British
people are becoming more and more literary. But I would point out that
literature by no means comprises the whole field of knowledge, and that
the disturbing thirst to improve one's self--to increase one's
knowledge--may well be slaked quite apart from literature.  With the
various ways of slaking I shall deal later. Here I merely point out to
those who have no natural sympathy with literature that literature is
not the only well.


\addtocounter{chapter}{1}\chapter*{Precautions Before Beginning}
\markright{\MakeUppercase{\romannumeral\thechapter} Precautions Before Beginning}
\addcontentsline{toc}{chapter}{~~~~~~~~Precautions Before Beginning}

Now that I have succeeded (if succeeded I have) in persuading you to
admit to yourself that you are constantly haunted by a suppressed
dissatisfaction with your own arrangement of your daily life; and that
the primal cause of that inconvenient dissatisfaction is the feeling
that you are every day leaving undone something which you would like to
do, and which, indeed, you are always hoping to do when you have "more
time"; and now that I have drawn your attention to the glaring,
dazzling truth that you never will have "more time," since you already
have all the time there is--you expect me to let you into some
wonderful secret by which you may at any rate approach the ideal of a
perfect arrangement of the day, and by which, therefore, that haunting,
unpleasant, daily disappointment of things left undone will be got rid
of!

I have found no such wonderful secret.  Nor do I expect to find it, nor
do I expect that anyone else will ever find it.  It is undiscovered.
When you first began to gather my drift, perhaps there was a
resurrection of hope in your breast.  Perhaps you said to yourself,
"This man will show me an easy, unfatiguing way of doing what I have so
long in vain wished to do."  Alas, no!  The fact is that there is no
easy way, no royal road.  The path to Mecca is extremely hard and
stony, and the worst of it is that you never quite get there after all.

The most important preliminary to the task of arranging one's life so
that one may live fully and comfortably within one's daily budget of
twenty-four hours is the calm realisation of the extreme difficulty of
the task, of the sacrifices and the endless effort which it demands.  I
cannot too strongly insist on this.

If you imagine that you will be able to achieve your ideal by
ingeniously planning out a time-table with a pen on a piece of paper,
you had better give up hope at once.  If you are not prepared for
discouragements and disillusions; if you will not be content with a
small result for a big effort, then do not begin.  Lie down again and
resume the uneasy doze which you call your existence.

It is very sad, is it not, very depressing and sombre?  And yet I think
it is rather fine, too, this necessity for the tense bracing of the
will before anything worth doing can be done.  I rather like it myself.
I feel it to be the chief thing that differentiates me from the cat by
the fire.

"Well," you say, "assume that I am braced for the battle.  Assume that
I have carefully weighed and comprehended your ponderous remarks; how
do I begin?"  Dear sir, you simply begin.  There is no magic method of
beginning.  If a man standing on the edge of a swimming-bath and
wanting to jump into the cold water should ask you, "How do I begin to
jump?" you would merely reply, "Just jump. Take hold of your nerves,
and jump."

As I have previously said, the chief beauty about the constant supply
of time is that you cannot waste it in advance.  The next year, the
next day, the next hour are lying ready for you, as perfect, as
unspoilt, as if you had never wasted or misapplied a single moment in
all your career.  Which fact is very gratifying and reassuring.  You
can turn over a new leaf every hour if you choose. Therefore no object
is served in waiting till next week, or even until to-morrow.  You may
fancy that the water will be warmer next week.  It won't.  It will be
colder.

But before you begin, let me murmur a few words of warning in your
private ear.

Let me principally warn you against your own ardour.  Ardour in
well-doing is a misleading and a treacherous thing.  It cries out
loudly for employment; you can't satisfy it at first; it wants more and
more; it is eager to move mountains and divert the course of rivers.
It isn't content till it perspires. And then, too often, when it feels
the perspiration on its brow, it wearies all of a sudden and dies,
without even putting itself to the trouble of saying, "I've had enough
of this."

Beware of undertaking too much at the start.  Be content with quite a
little. Allow for accidents.  Allow for human nature, especially your
own.

A failure or so, in itself, would not matter, if it did not incur a
loss of self-esteem and of self-confidence.  But just as nothing
succeeds like success, so nothing fails like failure.  Most people who
are ruined are ruined by attempting too much.  Therefore, in setting
out on the immense enterprise of living fully and comfortably within
the narrow limits of twenty-four hours a day, let us avoid at any cost
the risk of an early failure.  I will not agree that, in this business
at any rate, a glorious failure is better than a petty success.  I am
all for the petty success.  A glorious failure leads to nothing; a
petty success may lead to a success that is not petty.

So let us begin to examine the budget of the day's time. You say your
day is already full to overflowing.  How?  You actually spend in
earning your livelihood--how much?  Seven hours, on the average? And in
actual sleep, seven?  I will add two hours, and be generous. And I will
defy you to account to me on the spur of the moment for the other eight
hours.


\addtocounter{chapter}{1}\chapter*{The Cause of Troubles}
\markright{\MakeUppercase{\romannumeral\thechapter} The Cause of Troubles}
\addcontentsline{toc}{chapter}{~~~~~~~~The Cause of Troubles}

In order to come to grips at once with the question of time-expenditure
in all its actuality, I must choose an individual case for examination.
I can only deal with one case, and that case cannot be the average
case, because there is no such case as the average case, just as there
is no such man as the average man. Every man and every man's case is
special.

But if I take the case of a Londoner who works in an office, whose
office hours are from ten to six, and who spends fifty minutes morning
and night in travelling between his house door and his office door, I
shall have got as near to the average as facts permit.  There are men
who have to work longer for a living, but there are others who do not
have to work so long.

Fortunately the financial side of existence does not interest us here;
for our present purpose the clerk at a pound a week is exactly as well
off as the millionaire in Carlton House-terrace.

Now the great and profound mistake which my typical man makes in regard
to his day is a mistake of general attitude, a mistake which vitiates
and weakens two-thirds of his energies and interests.  In the majority
of instances he does not precisely feel a passion for his business; at
best he does not dislike it.  He begins his business functions with
reluctance, as late as he can, and he ends them with joy, as early as
he can.  And his engines while he is engaged in his business are seldom
at their full "h.p."  (I know that I shall be accused by angry readers
of traducing the city worker; but I am pretty thoroughly acquainted
with the City, and I stick to what I say.)

Yet in spite of all this he persists in looking upon those hours from
ten to six as "the day," to which the ten hours preceding them and the
six hours following them are nothing but a prologue and epilogue.  Such
an attitude, unconscious though it be, of course kills his interest in
the odd sixteen hours, with the result that, even if he does not waste
them, he does not count them; he regards them simply as margin.

This general attitude is utterly illogical and unhealthy, since it
formally gives the central prominence to a patch of time and a bunch of
activities which the man's one idea is to "get through" and have "done
with."  If a man makes two-thirds of his existence subservient to
one-third, for which admittedly he has no absolutely feverish zest, how
can he hope to live fully and completely?  He cannot.

If my typical man wishes to live fully and completely he must, in his
mind, arrange a day within a day.  And this inner day, a Chinese box in
a larger Chinese box, must begin at 6 p.m. and end at 10 a.m. It is a
day of sixteen hours; and during all these sixteen hours he has nothing
whatever to do but cultivate his body and his soul and his fellow men.
During those sixteen hours he is free; he is not a wage-earner; he is
not preoccupied with monetary cares; he is just as good as a man with a
private income.  This must be his attitude. And his attitude is all
important.  His success in life (much more important than the amount of
estate upon what his executors will have to pay estate duty) depends on
it.

What?  You say that full energy given to those sixteen hours will
lessen the value of the business eight?  Not so.  On the contrary, it
will assuredly increase the value of the business eight.  One of the
chief things which my typical man has to learn is that the mental
faculties are capable of a continuous hard activity; they do not tire
like an arm or a leg.  All they want is change--not rest, except in
sleep.

I shall now examine the typical man's current method of employing the
sixteen hours that are entirely his, beginning with his uprising.  I
will merely indicate things which he does and which I think he ought
not to do, postponing my suggestions for "planting" the times which I
shall have cleared--as a settler clears spaces in a forest.

In justice to him I must say that he wastes very little time before he
leaves the house in the morning at 9.10.  In too many houses he gets up
at nine, breakfasts between 9.7 and 9.9 1/2, and then bolts. But
immediately he bangs the front door his mental faculties, which are
tireless, become idle.  He walks to the station in a condition of
mental coma.  Arrived there, he usually has to wait for the train.  On
hundreds of suburban stations every morning you see men calmly
strolling up and down platforms while railway companies unblushingly
rob them of time, which is more than money.  Hundreds of thousands of
hours are thus lost every day simply because my typical man thinks so
little of time that it has never occurred to him to take quite easy
precautions against the risk of its loss.

He has a solid coin of time to spend every day--call it a sovereign. He
must get change for it, and in getting change he is content to lose
heavily.

Supposing that in selling him a ticket the company said, "We will
change you a sovereign, but we shall charge you three halfpence for
doing so," what would my typical man exclaim?  Yet that is the
equivalent of what the company does when it robs him of five minutes
twice a day.

You say I am dealing with minutiae.  I am.  And later on I will justify
myself.

Now will you kindly buy your paper and step into the train?


\addtocounter{chapter}{1}\chapter*{Tennis and the Immortal Soul}
\markright{\MakeUppercase{\romannumeral\thechapter} Tennis and the Immortal Soul}
\addcontentsline{toc}{chapter}{~~~~~~~~Tennis and the Immortal Soul}

You get into the morning train with your newspaper, and you calmly and
majestically give yourself up to your newspaper.  You do not hurry.
You know you have at least half an hour of security in front of you.
As your glance lingers idly at the advertisements of shipping and of
songs on the outer pages, your air is the air of a leisured man,
wealthy in time, of a man from some planet where there are a hundred
and twenty-four hours a day instead of twenty-four.  I am an
impassioned reader of newspapers. I read five English and two French
dailies, and the news-agents alone know how many weeklies, regularly.
I am obliged to mention this personal fact lest I should be accused of
a prejudice against newspapers when I say that I object to the reading
of newspapers in the morning train. Newspapers are produced with
rapidity, to be read with rapidity. There is no place in my daily
programme for newspapers.  I read them as I may in odd moments. But I
do read them.  The idea of devoting to them thirty or forty consecutive
minutes of wonderful solitude (for nowhere can one more perfectly
immerse one's self in one's self than in a compartment full of silent,
withdrawn, smoking males) is to me repugnant.  I cannot possibly allow
you to scatter priceless pearls of time with such Oriental lavishness.
You are not the Shah of time. Let me respectfully remind you that you
have no more time than I have.  No newspaper reading in trains!  I have
already "put by" about three-quarters of an hour for use.

Now you reach your office.  And I abandon you there till six o'clock.
I am aware that you have nominally an hour (often in reality an hour
and a half) in the midst of the day, less than half of which time is
given to eating.  But I will leave you all that to spend as you choose.
You may read your newspapers then.

I meet you again as you emerge from your office.  You are pale and
tired. At any rate, your wife says you are pale, and you give her to
understand that you are tired.  During the journey home you have been
gradually working up the tired feeling.  The tired feeling hangs heavy
over the mighty suburbs of London like a virtuous and melancholy cloud,
particularly in winter.  You don't eat immediately on your arrival
home. But in about an hour or so you feel as if you could sit up and
take a little nourishment.  And you do.  Then you smoke, seriously; you
see friends; you potter; you play cards; you flirt with a book; you
note that old age is creeping on; you take a stroll; you caress the
piano.... By Jove! a quarter past eleven. You then devote quite forty
minutes to thinking about going to bed; and it is conceivable that you
are acquainted with a genuinely good whisky.  At last you go to bed,
exhausted by the day's work.  Six hours, probably more, have gone since
you left the office--gone like a dream, gone like magic, unaccountably
gone!

That is a fair sample case.  But you say:  "It's all very well for you
to talk. A man \textit{is} tired.  A man must see his friends.  He can't
always be on the stretch."  Just so.  But when you arrange to go to the
theatre (especially with a pretty woman) what happens? You rush to the
suburbs; you spare no toil to make yourself glorious in fine raiment;
you rush back to town in another train; you keep yourself on the
stretch for four hours, if not five; you take her home; you take
yourself home.  You don't spend three-quarters of an hour in "thinking
about" going to bed.  You go.  Friends and fatigue have equally been
forgotten, and the evening has seemed so exquisitely long (or perhaps
too short)!  And do you remember that time when you were persuaded to
sing in the chorus of the amateur operatic society, and slaved two
hours every other night for three months?  Can you deny that when you
have something definite to look forward to at eventide, something that
is to employ all your energy--the thought of that something gives a
glow and a more intense vitality to the whole day?

What I suggest is that at six o'clock you look facts in the face and
admit that you are not tired (because you are not, you know), and that
you arrange your evening so that it is not cut in the middle by a meal.
By so doing you will have a clear expanse of at least three hours.  I
do not suggest that you should employ three hours every night of your
life in using up your mental energy. But I do suggest that you might,
for a commencement, employ an hour and a half every other evening in
some important and consecutive cultivation of the mind.  You will still
be left with three evenings for friends, bridge, tennis, domestic
scenes, odd reading, pipes, gardening, pottering, and prize
competitions.  You will still have the terrific wealth of forty-five
hours between 2 p.m. Saturday and 10 a.m. Monday.  If you persevere you
will soon want to pass four evenings, and perhaps five, in some
sustained endeavour to be genuinely alive. And you will fall out of
that habit of muttering to yourself at 11.15 p.m., "Time to be thinking
about going to bed."  The man who begins to go to bed forty minutes
before he opens his bedroom door is bored; that is to say, he is not
living.

But remember, at the start, those ninety nocturnal minutes thrice a
week must be the most important minutes in the ten thousand and eighty.
They must be sacred, quite as sacred as a dramatic rehearsal or a
tennis match. Instead of saying, "Sorry I can't see you, old chap, but
I have to run off to the tennis club," you must say, "...but I have to
work."  This, I admit, is intensely difficult to say.  Tennis is so
much more urgent than the immortal soul.



\addtocounter{chapter}{1}\chapter*{Remember Human Nature}
\markright{\MakeUppercase{\romannumeral\thechapter} Remember Human Nature}
\addcontentsline{toc}{chapter}{~~~~~~~~Remember Human Nature}

I have incidentally mentioned the vast expanse of forty-four hours
between leaving business at 2 p.m. on Saturday and returning to
business at 10 a.m. on Monday.  And here I must touch on the point
whether the week should consist of six days or of seven.  For many
years--in fact, until I was approaching forty--my own week consisted of
seven days.  I was constantly being informed by older and wiser people
that more work, more genuine living, could be got out of six days than
out of seven.

And it is certainly true that now, with one day in seven in which I
follow no programme and make no effort save what the caprice of the
moment dictates, I appreciate intensely the moral value of a weekly
rest.  Nevertheless, had I my life to arrange over again, I would do
again as I have done.  Only those who have lived at the full stretch
seven days a week for a long time can appreciate the full beauty of a
regular recurring idleness.  Moreover, I am ageing.  And it is a
question of age.  In cases of abounding youth and exceptional energy
and desire for effort I should say unhesitatingly:  Keep going, day in,
day out.

But in the average case I should say:  Confine your formal programme
(super-programme, I mean) to six days a week.  If you find yourself
wishing to extend it, extend it, but only in proportion to your wish;
and count the time extra as a windfall, not as regular income, so that
you can return to a six-day programme without the sensation of being
poorer, of being a backslider.

Let us now see where we stand.  So far we have marked for saving out of
the waste of days, half an hour at least on six mornings a week, and
one hour and a half on three evenings a week.  Total, seven hours and a
half a week.

I propose to be content with that seven hours and a half for the
present.  "What?" you cry.  "You pretend to show us how to live, and
you only deal with seven hours and a half out of a hundred and
sixty-eight!  Are you going to perform a miracle with your seven hours
and a half?"  Well, not to mince the matter, I am--if you will kindly
let me!  That is to say, I am going to ask you to attempt an experience
which, while perfectly natural and explicable, has all the air of a
miracle.  My contention is that the full use of those seven-and-a-half
hours will quicken the whole life of the week, add zest to it, and
increase the interest which you feel in even the most banal
occupations.  You practise physical exercises for a mere ten minutes
morning and evening, and yet you are not astonished when your physical
health and strength are beneficially affected every hour of the day,
and your whole physical outlook changed.  Why should you be astonished
that an average of over an hour a day given to the mind should
permanently and completely enliven the whole activity of the mind?

More time might assuredly be given to the cultivation of one's self.
And in proportion as the time was longer the results would be greater.
But I prefer to begin with what looks like a trifling effort.

It is not really a trifling effort, as those will discover who have yet
to essay it.  To "clear" even seven hours and a half from the jungle is
passably difficult.  For some sacrifice has to be made. One may have
spent one's time badly, but one did spend it; one did do something with
it, however ill-advised that something may have been.  To do something
else means a change of habits.

And habits are the very dickens to change!  Further, any change, even a
change for the better, is always accompanied by drawbacks and
discomforts.  If you imagine that you will be able to devote seven
hours and a half a week to serious, continuous effort, and still live
your old life, you are mistaken.  I repeat that some sacrifice, and an
immense deal of volition, will be necessary.  And it is because I know
the difficulty, it is because I know the almost disastrous effect of
failure in such an enterprise, that I earnestly advise a very humble
beginning. You must safeguard your self-respect. Self-respect is at the
root of all purposefulness, and a failure in an enterprise deliberately
planned deals a desperate wound at one's self-respect.  Hence I iterate
and reiterate: Start quietly, unostentatiously.

When you have conscientiously given seven hours and a half a week to
the cultivation of your vitality for three months--then you may begin
to sing louder and tell yourself what wondrous things you are capable
of doing.

Before coming to the method of using the indicated hours, I have one
final suggestion to make.  That is, as regards the evenings, to allow
much more than an hour and a half in which to do the work of an hour
and a half.  Remember the chance of accidents.  Remember human nature.
And give yourself, say, from 9 to 11.30 for your task of ninety minutes.


\addtocounter{chapter}{1}\chapter*{Controlling the Mind}
\markright{\MakeUppercase{\romannumeral\thechapter} Controlling the Mind}
\addcontentsline{toc}{chapter}{~~~~~~~~Controlling the Mind}

People say:  "One can't help one's thoughts."  But one can.  The
control of the thinking machine is perfectly possible.  And since
nothing whatever happens to us outside our own brain; since nothing
hurts us or gives us pleasure except within the brain, the supreme
importance of being able to control what goes on in that mysterious
brain is patent.  This idea is one of the oldest platitudes, but it is
a platitude whose profound truth and urgency most people live and die
without realising.  People complain of the lack of power to
concentrate, not witting that they may acquire the power, if they
choose.

And without the power to concentrate--that is to say, without the power
to dictate to the brain its task and to ensure obedience--true life is
impossible. Mind control is the first element of a full existence.

Hence, it seems to me, the first business of the day should be to put
the mind through its paces.  You look after your body, inside and out;
you run grave danger in hacking hairs off your skin; you employ a whole
army of individuals, from the milkman to the pig-killer, to enable you
to bribe your stomach into decent behaviour. Why not devote a little
attention to the far more delicate machinery of the mind, especially as
you will require no extraneous aid?  It is for this portion of the art
and craft of living that I have reserved the time from the moment of
quitting your door to the moment of arriving at your office.

"What?  I am to cultivate my mind in the street, on the platform, in
the train, and in the crowded street again?"  Precisely.  Nothing
simpler! No tools required!  Not even a book.  Nevertheless, the affair
is not easy.

When you leave your house, concentrate your mind on a subject (no
matter what, to begin with).  You will not have gone ten yards before
your mind has skipped away under your very eyes and is larking round
the corner with another subject.

Bring it back by the scruff of the neck.  Ere you have reached the
station you will have brought it back about forty times.  Do not
despair.  Continue. Keep it up.  You will succeed.  You cannot by any
chance fail if you persevere.  It is idle to pretend that your mind is
incapable of concentration. Do you not remember that morning when you
received a disquieting letter which demanded a very carefully-worded
answer?  How you kept your mind steadily on the subject of the answer,
without a second's intermission, until you reached your office;
whereupon you instantly sat down and wrote the answer?  That was a case
in which \textit{you} were roused by circumstances to such a degree of
vitality that you were able to dominate your mind like a tyrant. You
would have no trifling.  You insisted that its work should be done, and
its work was done.

By the regular practice of concentration (as to which there is no
secret--save the secret of perseverance) you can tyrannise over your
mind (which is not the highest part of \textit{you}) every hour of the day,
and in no matter what place.  The exercise is a very convenient one.
If you got into your morning train with a pair of dumb-bells for your
muscles or an encyclopaedia in ten volumes for your learning, you would
probably excite remark.  But as you walk in the street, or sit in the
corner of the compartment behind a pipe, or "strap-hang" on the
Subterranean, who is to know that you are engaged in the most important
of daily acts?  What asinine boor can laugh at you?

I do not care what you concentrate on, so long as you concentrate. It
is the mere disciplining of the thinking machine that counts. But
still, you may as well kill two birds with one stone, and concentrate
on something useful.  I suggest--it is only a suggestion--a little
chapter of Marcus Aurelius or Epictetus.

Do not, I beg, shy at their names.  For myself, I know nothing more
"actual," more bursting with plain common-sense, applicable to the
daily life of plain persons like you and me (who hate airs, pose, and
nonsense) than Marcus Aurelius or Epictetus.  Read a chapter--and so
short they are, the chapters!--in the evening and concentrate on it the
next morning.  You will see.

Yes, my friend, it is useless for you to try to disguise the fact. I
can hear your brain like a telephone at my ear.  You are saying to
yourself:  "This fellow was doing pretty well up to his seventh
chapter.  He had begun to interest me faintly.  But what he says about
thinking in trains, and concentration, and so on, is not for me.  It
may be well enough for some folks, but it isn't in my line."

It is for you, I passionately repeat; it is for you.  Indeed, you are
the very man I am aiming at.

Throw away the suggestion, and you throw away the most precious
suggestion that was ever offered to you.  It is not my suggestion. It
is the suggestion of the most sensible, practical, hard-headed men who
have walked the earth.  I only give it you at second-hand. Try it.  Get
your mind in hand.  And see how the process cures half the evils of
life--especially worry, that miserable, avoidable, shameful
disease--worry!


\addtocounter{chapter}{1}\chapter*{The Reflective Mood}
\markright{\MakeUppercase{\romannumeral\thechapter} The Reflective Mood}
\addcontentsline{toc}{chapter}{~~~~~~~~The Reflective Mood}

The exercise of concentrating the mind (to which at least half an hour
a day should be given) is a mere preliminary, like scales on the piano.
Having acquired power over that most unruly member of one's complex
organism, one has naturally to put it to the yoke. Useless to possess
an obedient mind unless one profits to the furthest possible degree by
its obedience.  A prolonged primary course of study is indicated.

Now as to what this course of study should be there cannot be any
question; there never has been any question.  All the sensible people
of all ages are agreed upon it.  And it is not literature, nor is it
any other art, nor is it history, nor is it any science. It is the
study of one's self.  Man, know thyself.  These words are so hackneyed
that verily I blush to write them. Yet they must be written, for they
need to be written.  (I take back my blush, being ashamed of it.)  Man,
know thyself.  I say it out loud.  The phrase is one of those phrases
with which everyone is familiar, of which everyone acknowledges the
value, and which only the most sagacious put into practice.  I don't
know why.  I am entirely convinced that what is more than anything else
lacking in the life of the average well-intentioned man of to-day is
the reflective mood.

We do not reflect.  I mean that we do not reflect upon genuinely
important things; upon the problem of our happiness, upon the main
direction in which we are going, upon what life is giving to us, upon
the share which reason has (or has not) in determining our actions, and
upon the relation between our principles and our conduct.

And yet you are in search of happiness, are you not?  Have you
discovered it?

The chances are that you have not.  The chances are that you have
already come to believe that happiness is unattainable.  But men have
attained it. And they have attained it by realising that happiness does
not spring from the procuring of physical or mental pleasure, but from
the development of reason and the adjustment of conduct to principles.

I suppose that you will not have the audacity to deny this.  And if you
admit it, and still devote no part of your day to the deliberate
consideration of your reason, principles, and conduct, you admit also
that while striving for a certain thing you are regularly leaving
undone the one act which is necessary to the attainment of that thing.

Now, shall I blush, or will you?

Do not fear that I mean to thrust certain principles upon your
attention.  I care not (in this place) what your principles are. Your
principles may induce you to believe in the righteousness of burglary.
I don't mind.  All I urge is that a life in which conduct does not
fairly well accord with principles is a silly life; and that conduct
can only be made to accord with principles by means of daily
examination, reflection, and resolution.  What leads to the permanent
sorrowfulness of burglars is that their principles are contrary to
burglary.  If they genuinely believed in the moral excellence of
burglary, penal servitude would simply mean so many happy years for
them; all martyrs are happy, because their conduct and their principles
agree.

As for reason (which makes conduct, and is not unconnected with the
making of principles), it plays a far smaller part in our lives than we
fancy.  We are supposed to be reasonable but we are much more
instinctive than reasonable. And the less we reflect, the less
reasonable we shall be.  The next time you get cross with the waiter
because your steak is over-cooked, ask reason to step into the
cabinet-room of your mind, and consult her.  She will probably tell you
that the waiter did not cook the steak, and had no control over the
cooking of the steak; and that even if he alone was to blame, you
accomplished nothing good by getting cross; you merely lost your
dignity, looked a fool in the eyes of sensible men, and soured the
waiter, while producing no effect whatever on the steak.

The result of this consultation with reason (for which she makes no
charge) will be that when once more your steak is over-cooked you will
treat the waiter as a fellow-creature, remain quite calm in a kindly
spirit, and politely insist on having a fresh steak.  The gain will be
obvious and solid.

In the formation or modification of principles, and the practice of
conduct, much help can be derived from printed books (issued at
sixpence each and upwards).  I mentioned in my last chapter Marcus
Aurelius and Epictetus. Certain even more widely known works will occur
at once to the memory. I may also mention Pascal, La Bruyere, and
Emerson.  For myself, you do not catch me travelling without my Marcus
Aurelius.  Yes, books are valuable.  But not reading of books will take
the place of a daily, candid, honest examination of what one has
recently done, and what one is about to do--of a steady looking at
one's self in the face (disconcerting though the sight may be).

When shall this important business be accomplished?  The solitude of
the evening journey home appears to me to be suitable for it.  A
reflective mood naturally follows the exertion of having earned the
day's living. Of course if, instead of attending to an elementary and
profoundly important duty, you prefer to read the paper (which you
might just as well read while waiting for your dinner) I have nothing
to say.  But attend to it at some time of the day you must. I now come
to the evening hours.



\addtocounter{chapter}{1}\chapter*{Interest in the Arts}
\markright{\MakeUppercase{\romannumeral\thechapter} Interest in the Arts}
\addcontentsline{toc}{chapter}{~~~~~~~~Interest in the Arts}

Many people pursue a regular and uninterrupted course of idleness in
the evenings because they think that there is no alternative to
idleness but the study of literature; and they do not happen to have a
taste for literature.  This is a great mistake.

Of course it is impossible, or at any rate very difficult, properly to
study anything whatever without the aid of printed books.  But if you
desire to understand the deeper depths of bridge or of boat-sailing you
would not be deterred by your lack of interest in literature from
reading the best books on bridge or boat-sailing. We must, therefore,
distinguish between literature, and books treating of subjects not
literary.  I shall come to literature in due course.

Let me now remark to those who have never read Meredith, and who are
capable of being unmoved by a discussion as to whether Mr. Stephen
Phillips is or is not a true poet, that they are perfectly within their
rights. It is not a crime not to love literature.  It is not a sign of
imbecility.  The mandarins of literature will order out to instant
execution the unfortunate individual who does not comprehend, say, the
influence of Wordsworth on Tennyson.  But that is only their impudence.
Where would they be, I wonder, if requested to explain the influences
that went to make Tschaikowsky's "Pathetic Symphony"?

There are enormous fields of knowledge quite outside literature which
will yield magnificent results to cultivators.  For example (since I
have just mentioned the most popular piece of high-class music in
England to-day), I am reminded that the Promenade Concerts begin in
August. You go to them.  You smoke your cigar or cigarette (and I
regret to say that you strike your matches during the soft bars of the
"Lohengrin" overture), and you enjoy the music.  But you say you cannot
play the piano or the fiddle, or even the banjo; that you know nothing
of music.

What does that matter?  That you have a genuine taste for music is
proved by the fact that, in order to fill his hall with you and your
peers, the conductor is obliged to provide programmes from which bad
music is almost entirely excluded (a change from the old Covent Garden
days!).

Now surely your inability to perform "The Maiden's Prayer" on a piano
need not prevent you from making yourself familiar with the
construction of the orchestra to which you listen a couple of nights a
week during a couple of months!  As things are, you probably think of
the orchestra as a heterogeneous mass of instruments producing a
confused agreeable mass of sound.  You do not listen for details
because you have never trained your ears to listen to details.

If you were asked to name the instruments which play the great theme at
the beginning of the C minor symphony you could not name them for your
life's sake.  Yet you admire the C minor symphony.  It has thrilled
you.  It will thrill you again.  You have even talked about it, in an
expansive mood, to that lady--you know whom I mean.  And all you can
positively state about the C minor symphony is that Beethoven composed
it and that it is a "jolly fine thing."

Now, if you have read, say, Mr. Krehbiel's "How to Listen to Music"
(which can be got at any bookseller's for less than the price of a
stall at the Alhambra, and which contains photographs of all the
orchestral instruments and plans of the arrangement of orchestras) you
would next go to a promenade concert with an astonishing
intensification of interest in it.  Instead of a confused mass, the
orchestra would appear to you as what it is--a marvellously balanced
organism whose various groups of members each have a different and an
indispensable function.  You would spy out the instruments, and listen
for their respective sounds.  You would know the gulf that separates a
French horn from an English horn, and you would perceive why a player
of the hautboy gets higher wages than a fiddler, though the fiddle is
the more difficult instrument.  You would \textit{live} at a promenade
concert, whereas previously you had merely existed there in a state of
beatific coma, like a baby gazing at a bright object.

The foundations of a genuine, systematic knowledge of music might be
laid. You might specialise your inquiries either on a particular form
of music (such as the symphony), or on the works of a particular
composer.  At the end of a year of forty-eight weeks of three brief
evenings each, combined with a study of programmes and attendances at
concerts chosen out of your increasing knowledge, you would really know
something about music, even though you were as far off as ever from
jangling "The Maiden's Prayer" on the piano.

"But I hate music!" you say.  My dear sir, I respect you.

What applies to music applies to the other arts.  I might mention Mr.
Clermont Witt's "How to Look at Pictures," or Mr. Russell Sturgis's
"How to Judge Architecture," as beginnings (merely beginnings) of
systematic vitalising knowledge in other arts, the materials for whose
study abound in London.

"I hate all the arts!" you say.  My dear sir, I respect you more and
more.

I will deal with your case next, before coming to literature.



\addtocounter{chapter}{1}\chapter*{Nothing in Life is Humdrum}
\markright{\MakeUppercase{\romannumeral\thechapter} Nothing in Life is Humdrum}
\addcontentsline{toc}{chapter}{~~~~~~~~Nothing in Life is Humdrum}

Art is a great thing.  But it is not the greatest.  The most important
of all perceptions is the continual perception of cause and effect--in
other words, the perception of the continuous development of the
universe--in still other words, the perception of the course of
evolution.  When one has thoroughly got imbued into one's head the
leading truth that nothing happens without a cause, one grows not only
large-minded, but large-hearted.

It is hard to have one's watch stolen, but one reflects that the thief
of the watch became a thief from causes of heredity and environment
which are as interesting as they are scientifically comprehensible; and
one buys another watch, if not with joy, at any rate with a philosophy
that makes bitterness impossible.  One loses, in the study of cause and
effect, that absurd air which so many people have of being always
shocked and pained by the curiousness of life.  Such people live amid
human nature as if human nature were a foreign country full of awful
foreign customs. But, having reached maturity, one ought surely to be
ashamed of being a stranger in a strange land!

The study of cause and effect, while it lessens the painfulness of
life, adds to life's picturesqueness.  The man to whom evolution is but
a name looks at the sea as a grandiose, monotonous spectacle, which he
can witness in August for three shillings third-class return.  The man
who is imbued with the idea of development, of continuous cause and
effect, perceives in the sea an element which in the
day-before-yesterday of geology was vapour, which yesterday was
boiling, and which to-morrow will inevitably be ice.

He perceives that a liquid is merely something on its way to be solid,
and he is penetrated by a sense of the tremendous, changeful
picturesqueness of life.  Nothing will afford a more durable
satisfaction than the constantly cultivated appreciation of this. It is
the end of all science.

Cause and effect are to be found everywhere.  Rents went up in
Shepherd's Bush.  It was painful and shocking that rents should go up
in Shepherd's Bush.  But to a certain point we are all scientific
students of cause and effect, and there was not a clerk lunching at a
Lyons Restaurant who did not scientifically put two and two together
and see in the (once) Two-penny Tube the cause of an excessive demand
for wigwams in Shepherd's Bush, and in the excessive demand for wigwams
the cause of the increase in the price of wigwams.

"Simple!" you say, disdainfully.  Everything--the whole complex
movement of the universe--is as simple as that--when you can
sufficiently put two and two together.  And, my dear sir, perhaps you
happen to be an estate agent's clerk, and you hate the arts, and you
want to foster your immortal soul, and you can't be interested in your
business because it's so humdrum.

Nothing is humdrum.

The tremendous, changeful picturesqueness of life is marvellously shown
in an estate agent's office.  What!  There was a block of traffic in
Oxford Street; to avoid the block people actually began to travel under
the cellars and drains, and the result was a rise of rents in
Shepherd's Bush!  And you say that isn't picturesque! Suppose you were
to study, in this spirit, the property question in London for an hour
and a half every other evening. Would it not give zest to your
business, and transform your whole life?

You would arrive at more difficult problems.  And you would be able to
tell us why, as the natural result of cause and effect, the longest
straight street in London is about a yard and a half in length, while
the longest absolutely straight street in Paris extends for miles.  I
think you will admit that in an estate agent's clerk I have not chosen
an example that specially favours my theories.

You are a bank clerk, and you have not read that breathless romance
(disguised as a scientific study), Walter Bagehot's "Lombard Street"?
Ah, my dear sir, if you had begun with that, and followed it up for
ninety minutes every other evening, how enthralling your business would
be to you, and how much more clearly you would understand human nature.

You are "penned in town," but you love excursions to the country and
the observation of wild life--certainly a heart-enlarging diversion.
Why don't you walk out of your house door, in your slippers, to the
nearest gas lamp of a night with a butterfly net, and observe the wild
life of common and rare moths that is beating about it, and co-ordinate
the knowledge thus obtained and build a superstructure on it, and at
last get to know something about something?

You need not be devoted to the arts, not to literature, in order to
live fully.

The whole field of daily habit and scene is waiting to satisfy that
curiosity which means life, and the satisfaction of which means an
understanding heart.

I promised to deal with your case, O man who hates art and literature,
and I have dealt with it.  I now come to the case of the person,
happily very common, who does "like reading."



\addtocounter{chapter}{1}\chapter*{Serious Reading}
\markright{\MakeUppercase{\romannumeral\thechapter} Serious Reading}
\addcontentsline{toc}{chapter}{~~~~~~~~Serious Reading}

Novels are excluded from "serious reading," so that the man who, bent
on self-improvement, has been deciding to devote ninety minutes three
times a week to a complete study of the works of Charles Dickens will
be well advised to alter his plans.  The reason is not that novels are
not serious--some of the great literature of the world is in the form
of prose fiction--the reason is that bad novels ought not to be read,
and that good novels never demand any appreciable mental application on
the part of the reader. It is only the bad parts of Meredith's novels
that are difficult.  A good novel rushes you forward like a skiff down
a stream, and you arrive at the end, perhaps breathless, but
unexhausted.  The best novels involve the least strain.  Now in the
cultivation of the mind one of the most important factors is precisely
the feeling of strain, of difficulty, of a task which one part of you
is anxious to achieve and another part of you is anxious to shirk; and
that feeling cannot be got in facing a novel.  You do not set your
teeth in order to read "Anna Karenina."  Therefore, though you should
read novels, you should not read them in those ninety minutes.

Imaginative poetry produces a far greater mental strain than novels. It
produces probably the severest strain of any form of literature. It is
the highest form of literature.  It yields the highest form of
pleasure, and teaches the highest form of wisdom.  In a word, there is
nothing to compare with it.  I say this with sad consciousness of the
fact that the majority of people do not read poetry.

I am persuaded that many excellent persons, if they were confronted
with the alternatives of reading "Paradise Lost" and going round
Trafalgar Square at noonday on their knees in sack-cloth, would choose
the ordeal of public ridicule.  Still, I will never cease advising my
friends and enemies to read poetry before anything.

If poetry is what is called "a sealed book" to you, begin by reading
Hazlitt's famous essay on the nature of "poetry in general."  It is the
best thing of its kind in English, and no one who has read it can
possibly be under the misapprehension that poetry is a mediaeval
torture, or a mad elephant, or a gun that will go off by itself and
kill at forty paces. Indeed, it is difficult to imagine the mental
state of the man who, after reading Hazlitt's essay, is not urgently
desirous of reading some poetry before his next meal.  If the essay so
inspires you I would suggest that you make a commencement with purely
narrative poetry.

There is an infinitely finer English novel, written by a woman, than
anything by George Eliot or the Brontes, or even Jane Austen, which
perhaps you have not read.  Its title is "Aurora Leigh," and its author
E.B. Browning.  It happens to be written in verse, and to contain a
considerable amount of genuinely fine poetry.  Decide to read that book
through, even if you die for it.  Forget that it is fine poetry. Read
it simply for the story and the social ideas.  And when you have done,
ask yourself honestly whether you still dislike poetry. I have known
more than one person to whom "Aurora Leigh" has been the means of
proving that in assuming they hated poetry they were entirely mistaken.

Of course, if, after Hazlitt, and such an experiment made in the light
of Hazlitt, you are finally assured that there is something in you
which is antagonistic to poetry, you must be content with history or
philosophy. I shall regret it, yet not inconsolably. "The Decline and
Fall" is not to be named in the same day with "Paradise Lost," but it
is a vastly pretty thing; and Herbert Spencer's "First Principles"
simply laughs at the claims of poetry and refuses to be accepted as
aught but the most majestic product of any human mind.  I do not
suggest that either of these works is suitable for a tyro in mental
strains.  But I see no reason why any man of average intelligence
should not, after a year of continuous reading, be fit to assault the
supreme masterpieces of history or philosophy.  The great convenience
of masterpieces is that they are so astonishingly lucid.

I suggest no particular work as a start.  The attempt would be futile
in the space of my command.  But I have two general suggestions of a
certain importance.  The first is to define the direction and scope of
your efforts. Choose a limited period, or a limited subject, or a
single author.  Say to yourself:  "I will know something about the
French Revolution, or the rise of railways, or the works of John
Keats."  And during a given period, to be settled beforehand, confine
yourself to your choice.  There is much pleasure to be derived from
being a specialist.

The second suggestion is to think as well as to read.  I know people
who read and read, and for all the good it does them they might just as
well cut bread-and-butter.  They take to reading as better men take to
drink. They fly through the shires of literature on a motor-car, their
sole object being motion.  They will tell you how many books they have
read in a year.

Unless you give at least forty-five minutes to careful, fatiguing
reflection (it is an awful bore at first) upon what you are reading,
your ninety minutes of a night are chiefly wasted.  This means that
your pace will be slow.

Never mind.

Forget the goal; think only of the surrounding country; and after a
period, perhaps when you least expect it, you will suddenly find
yourself in a lovely town on a hill.



\addtocounter{chapter}{1}\chapter*{Dangers to Avoid}
\markright{\MakeUppercase{\romannumeral\thechapter} Dangers to Avoid}
\addcontentsline{toc}{chapter}{~~~~~~~~Dangers to Avoid}

I cannot terminate these hints, often, I fear, too didactic and abrupt,
upon the full use of one's time to the great end of living (as
distinguished from vegetating) without briefly referring to certain
dangers which lie in wait for the sincere aspirant towards life.  The
first is the terrible danger of becoming that most odious and least
supportable of persons--a prig. Now a prig is a pert fellow who gives
himself airs of superior wisdom. A prig is a pompous fool who has gone
out for a ceremonial walk, and without knowing it has lost an important
part of his attire, namely, his sense of humour.  A prig is a tedious
individual who, having made a discovery, is so impressed by his
discovery that he is capable of being gravely displeased because the
entire world is not also impressed by it. Unconsciously to become a
prig is an easy and a fatal thing.

Hence, when one sets forth on the enterprise of using all one's time,
it is just as well to remember that one's own time, and not other
people's time, is the material with which one has to deal; that the
earth rolled on pretty comfortably before one began to balance a budget
of the hours, and that it will continue to roll on pretty comfortably
whether or not one succeeds in one's new role of chancellor of the
exchequer of time.  It is as well not to chatter too much about what
one is doing, and not to betray a too-pained sadness at the spectacle
of a whole world deliberately wasting so many hours out of every day,
and therefore never really living.  It will be found, ultimately, that
in taking care of one's self one has quite all one can do.

Another danger is the danger of being tied to a programme like a slave
to a chariot.  One's programme must not be allowed to run away with
one. It must be respected, but it must not be worshipped as a fetish.
A programme of daily employ is not a religion.

This seems obvious.  Yet I know men whose lives are a burden to
themselves and a distressing burden to their relatives and friends
simply because they have failed to appreciate the obvious.  "Oh, no," I
have heard the martyred wife exclaim, "Arthur always takes the dog out
for exercise at eight o'clock and he always begins to read at a quarter
to nine.  So it's quite out of the question that we should..." etc.,
etc.  And the note of absolute finality in that plaintive voice reveals
the unsuspected and ridiculous tragedy of a career.

On the other hand, a programme is a programme.  And unless it is
treated with deference it ceases to be anything but a poor joke.  To
treat one's programme with exactly the right amount of deference, to
live with not too much and not too little elasticity, is scarcely the
simple affair it may appear to the inexperienced.

And still another danger is the danger of developing a policy of rush,
of being gradually more and more obsessed by what one has to do next.
In this way one may come to exist as in a prison, and one's life may
cease to be one's own.  One may take the dog out for a walk at eight
o'clock, and meditate the whole time on the fact that one must begin to
read at a quarter to nine, and that one must not be late.

And the occasional deliberate breaking of one's programme will not help
to mend matters.  The evil springs not from persisting without
elasticity in what one has attempted, but from originally attempting
too much, from filling one's programme till it runs over.  The only
cure is to reconstitute the programme, and to attempt less.

But the appetite for knowledge grows by what it feeds on, and there are
men who come to like a constant breathless hurry of endeavour. Of them
it may be said that a constant breathless hurry is better than an
eternal doze.

In any case, if the programme exhibits a tendency to be oppressive, and
yet one wishes not to modify it, an excellent palliative is to pass
with exaggerated deliberation from one portion of it to another; for
example, to spend five minutes in perfect mental quiescence between
chaining up the St. Bernard and opening the book; in other words, to
waste five minutes with the entire consciousness of wasting them.

The last, and chiefest danger which I would indicate, is one to which I
have already referred--the risk of a failure at the commencement of the
enterprise.

I must insist on it.

A failure at the commencement may easily kill outright the newborn
impulse towards a complete vitality, and therefore every precaution
should be observed to avoid it.  The impulse must not be over-taxed.
Let the pace of the first lap be even absurdly slow, but let it be as
regular as possible.

And, having once decided to achieve a certain task, achieve it at all
costs of tedium and distaste.  The gain in self-confidence of having
accomplished a tiresome labour is immense.

Finally, in choosing the first occupations of those evening hours, be
guided by nothing whatever but your taste and natural inclination.

It is a fine thing to be a walking encyclopaedia of philosophy, but if
you happen to have no liking for philosophy, and to have a like for the
natural history of street-cries, much better leave philosophy alone,
and take to street-cries.

%%% Local Variables: 
%%% mode: latex
%%% TeX-master: "Wrapper"
%%% End: 
